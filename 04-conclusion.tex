\chapter*{Conclusion et perspectives}
\addcontentsline{toc}{chapter}{Conclusion}
\markboth{Conclusion}{Conclusion}
\label{sec:conclusion}

Au long de ces six mois de stage passés sur le site d'Orange de Sophia-Antipolis, j'ai pu découvrir un métier aux enjeux variés allant de l'explication de technologies en interne à 
l'explication de méthodes de travail qui ciblera des personnes en dehors de l'entreprise. 
J'ai particulièrement apprécié la liberté de création qui m'a été accordée pour la réalisation des vidéos, même si j'aurais aimé bénéficier de plus de temps pour pouvoir créer moi même la totalité des illustrations. J'ai aussi dû approfondir mes connaissances du logiciel After Effect utilisé pour le montage pour pouvoir produire des effets de transitions que nous n'avions pas eu le temps de réaliser dans les cours de pratiques plastiques. Ce stage m'aura permis de confirmer que les missions créatives sont les plus adaptées pour moi, mais qu'il ne faut pas non plus négliger l'aspect technique qui reste très présent.\\

La DFY m’a accueilli pendant une période charnière : son passage aux APIs via le cloud, et je suis fier d’avoir pu prendre part à ce changement qui va permettre une facilité d'intégration des applications et l'accès aux données ainsi qu'à la sécurisation des échanges qui est un paramètre plus qu'important de nos jours. J'ai notamment étoffé mes connaissances sur la sécurité et les étapes de créations d'une application qui change ne nos habitudes universitaires. En effet là où nous commençons chaque application à partir de rien, celles de la DFY sont construites à partir de briques pré-établies. Les connaissances acquises au cours de mon cursus m'ont permis de comprendre le fonctionnement de la plupart des composantes de l'écosystème PnS.\\

Suite à cette expérience, je souhaite m'orienter dans un secteur qui soit aussi technologique et innovant que celui dans lequel j'ai effectué ce stage et dans lequel il y aurait toujours une part importante de créativité.


%%% Local Variables: 
%%% mode: latex
%%% TeX-master: "isae-report-template"
%%% End: 


\chapter*{Introduction}
\addcontentsline{toc}{chapter}{Introduction}
\markboth{Introduction}{Introduction}
\label{chap:introduction}
%\minitoc

Avec 30 millions de clients mobile et plus de 18 millions de clients  pour le haut et très haut débit fixe, Orange est le premier opérateur en France et la 50\up{ième} entreprise la plus influente au monde en 2016. Avec une telle importance autant au niveau national qu'international, la firme se doit de rester innovante et à la pointe de la technologie pour pouvoir répondre aux besoins de plus en plus précis et diversifiés de leurs clients. La division que j’ai intégré, la Smart Data Factory (SDFY), opère sur plusieurs secteurs différents allant de la gestion de base de données à la recommandation personnalisée en passant par les application mobiles et la sécurité. Avec une telle diversité de secteurs d’activités, il n’est pas évident de tenir toutes les équipes au courant de toutes les nouveautés et avancées au sein de chaque projet ni de pouvoir les expliquer simplement et ludiquement aux personnes sans connaissances techniques.\\
<<<<<<< HEAD
L’offre de stage à laquelle j’ai répondu s’inscrit dans la lignée d’un projet qui avait été initié par un prestataire externe : pouvoir expliquer de manière compréhensible et divertissant sous forme de vidéos le fonctionnement et l’intérêt des nouvelles  interfaces de programmations (APIs) produites au sein de la SFDY dans le cadre du programme PnS.com, qui a pour but de fournir un stockage des données en libre service. J’ai choisi de répondre à cette offre car j’étais intéressé par le côté créatif de la réalisation de vidéos qui était une suite directe du cours de « Pratiques plastiques » que j’ai suivi cette année et qui m’a beaucoup intéressé. C’était aussi l’occasion pour moi de mettre en pratique les connaissances techniques acquises au cours de mes cinq années d'études à l’Université de Bourgogne.\\
=======
L’offre de stage à laquelle j’ai répondu s’inscrit dans la lignée d’un projet qui avait été initiée par un prestataire externe : pouvoir expliquer de manière compréhensible et divertissant sous forme de vidéos le fonctionnement et l’intérêt des nouvelles  interfaces de programmations (APIs) produites au sein de la SFDY dans le cadre du programme PnS.com, qui a pour but de fournir un stockage des données en libre service. J’ai choisi de répondre à cette offre car j’étais intéressé par le côté créatif de la réalisation de vidéos qui était une suite directe du cours de « Pratiques plastiques » que j’ai suivi cette année et qui m’a beaucoup plu. C’était aussi l’occasion pour moi de mettre en pratique les connaissances techniques acquises au cours de mes cinq années d'études à l’Université de Bourgogne.\\
>>>>>>> origin/master
Dans ce rapport, nous étudierons plus en détail l’entreprise et son secteur d’activité, puis nous nous concentrerons sur mon stage, en nous focalisant sur mes différentes missions puis mon bilan.
%%% Local Variables: 
%%% mode: latex
%%% TeX-master: "isae-report-template"
%%% End: 

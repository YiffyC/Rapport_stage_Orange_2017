
\documentclass[a4paper,12pt]{book}
\usepackage[utf8]{inputenc}
\usepackage[T1]{fontenc}
\usepackage[frenchb]{babel}
\usepackage{a4wide}
\usepackage{graphicx}
\graphicspath{{images/}}
\usepackage{subfig}
\usepackage{tikz}
\usetikzlibrary{shapes,arrows}
\usepackage{pgfplots}
\pgfplotsset{compat=newest}
\pgfplotsset{plot coordinates/math parser=false}
\newlength\figureheight
\newlength\figurewidth
\pgfkeys{/pgf/number format/.cd,
set decimal separator={,\!},
1000 sep={\,},
}
\usepackage{ifthen}
\usepackage{ifpdf}
\ifpdf
\usepackage[pdftex]{hyperref}
\else
\usepackage{hyperref}
\fi
\usepackage{color}
\hypersetup{%
colorlinks=true,
linkcolor=black,
citecolor=black,
urlcolor=black}

\renewcommand{\baselinestretch}{1.05}
\usepackage{fancyhdr}
\pagestyle{fancy}
\fancyfoot{}
\fancyhead[LE,RO]{\bfseries\thepage}
\fancyhead[RE]{\bfseries\nouppercase{\leftmark}}
\fancyhead[LO]{\bfseries\nouppercase{\rightmark}}
\setlength{\headheight}{15pt}

\let\headruleORIG\headrule
\renewcommand{\headrule}{\color{black} \headruleORIG}
\renewcommand{\headrulewidth}{1.0pt}
\usepackage{colortbl}
\arrayrulecolor{black}

\fancypagestyle{plain}{
  \fancyhead{}
  \fancyfoot[C]{\thepage}
  \renewcommand{\headrulewidth}{0pt}
}

\makeatletter
\def\@textbottom{\vskip \z@ \@plus 1pt}
\let\@texttop\relax
\makeatother

\makeatletter
\def\cleardoublepage{\clearpage\if@twoside \ifodd\c@page\else%
  \hbox{}%
  \thispagestyle{empty}%
  \newpage%
  \if@twocolumn\hbox{}\newpage\fi\fi\fi}
\makeatother

\usepackage{amsthm}
\usepackage{amssymb,amsmath,bbm}
\usepackage{array}
\usepackage{bm}
\usepackage{multirow}
\usepackage[footnote]{acronym}

\newcommand*{\SET}[1]  {\ensuremath{\mathbf{#1}}}
\newcommand*{\VEC}[1]  {\ensuremath{\boldsymbol{#1}}}
\newcommand*{\FAM}[1]  {\ensuremath{\boldsymbol{#1}}}
\newcommand*{\MAT}[1]  {\ensuremath{\boldsymbol{#1}}}
\newcommand*{\OP}[1]  {\ensuremath{\mathrm{#1}}}
\newcommand*{\NORM}[1]  {\ensuremath{\left\|#1\right\|}}
\newcommand*{\DPR}[2]  {\ensuremath{\left \langle #1,#2 \right \rangle}}
\newcommand*{\calbf}[1]  {\ensuremath{\boldsymbol{\mathcal{#1}}}}
\newcommand*{\shift}[1]  {\ensuremath{\boldsymbol{#1}}}

\newcommand{\eqdef}{\stackrel{\mathrm{def}}{=}}
\newcommand{\argmax}{\operatornamewithlimits{argmax}}
\newcommand{\argmin}{\operatornamewithlimits{argmin}}
\newcommand{\ud}{\, \mathrm{d}}
\newcommand{\vect}{\text{Vect}}
\newcommand{\sinc}{\ensuremath{\mathrm{sinc}}}
\newcommand{\esp}{\ensuremath{\mathbb{E}}}
\newcommand{\hilbert}{\ensuremath{\mathcal{H}}}
\newcommand{\fourier}{\ensuremath{\mathcal{F}}}
\newcommand{\sgn}{\text{sgn}}
\newcommand{\intTT}{\int_{-T}^{T}}
\newcommand{\intT}{\int_{-\frac{T}{2}}^{\frac{T}{2}}}
\newcommand{\intinf}{\int_{-\infty}^{+\infty}}
\newcommand{\Sh}{\ensuremath{\boldsymbol{S}}}
\newcommand{\C}{\SET{C}}
\newcommand{\R}{\SET{R}}
\newcommand{\Z}{\SET{Z}}
\newcommand{\N}{\SET{N}}
\newcommand{\K}{\SET{K}}
\newcommand{\reel}{\mathcal{R}}
\newcommand{\imag}{\mathcal{I}}
\newcommand{\cmnr}{c_{m,n}^\reel}
\newcommand{\cmni}{c_{m,n}^\imag}
\newcommand{\cnr}{c_{n}^\reel}
\newcommand{\cni}{c_{n}^\imag}
\newcommand{\tproto}{g}
\newcommand{\rproto}{\check{g}}
\newcommand{\LR}{\mathcal{L}_2(\SET{R})}
\newcommand{\LZ}{\ell_2(\SET{Z})}
\newcommand{\LZI}[1]{\ell_2(\SET{#1})}
\newcommand{\LZZ}{\ell_2(\SET{Z}^2)}
\newcommand{\diag}{\operatorname{diag}}
\newcommand{\noise}{z}
\newcommand{\Noise}{Z}
\newcommand{\filtnoise}{\zeta}
\newcommand{\tp}{g}
\newcommand{\rp}{\check{g}}
\newcommand{\TP}{G}
\newcommand{\RP}{\check{G}}
\newcommand{\dmin}{d_{\mathrm{min}}}
\newcommand{\Dmin}{D_{\mathrm{min}}}
\newcommand{\Image}{\ensuremath{\text{Im}}}
\newcommand{\Span}{\ensuremath{\text{Span}}}

\newtheoremstyle{break}
  {11pt}{11pt}%
  {\itshape}{}%
  {\bfseries}{}%
  {\newline}{}%
\theoremstyle{break}

%\theoremstyle{definition}
\newtheorem{definition}{Définition}[chapter]

%\theoremstyle{definition}
\newtheorem{theoreme}{Théorème}[chapter]

%\theoremstyle{remark}
\newtheorem{remarque}{Remarque}[chapter]

%\theoremstyle{plain}
\newtheorem{propriete}{Propriété}[chapter]
\newtheorem{exemple}{Exemple}[chapter]

\parskip=5pt
%\sloppy

\begin{document}

%%%%%%%%%%%%%%%%%%
%%% First page %%%
%%%%%%%%%%%%%%%%%%

\begin{titlepage}
\begin{center}

\includegraphics[width=0.6\textwidth]{entete}\\[1cm]

{\large Master 2 multimédia, Université de Bourgogne}\\[0.5cm]

{\large Rapport de stage de fin d'études}\\[0.5cm]

% Title
\rule{\linewidth}{0.5mm} \\[0.4cm]
{ \huge \bfseries Chargé de communication autour du projet PnS.com \\[0.4cm] }
\rule{\linewidth}{0.5mm} \\[1.5cm]

% Author and supervisor
\noindent
\begin{minipage}{0.4\textwidth}
  \begin{flushleft} \large
    \emph{Auteur :}\\
    Mr. Bérenger \textsc{Thévenet}\\
  \end{flushleft}
\end{minipage}%
\begin{minipage}{0.4\textwidth}
  \begin{flushright} \large
    \emph{Encadrants :} \\
    Mme.~Sandrine \textsc{Lanquetin}\\
    Mr.~Joel \textsc{Savelli}\\
    Mr.~Olivier \textsc{Leick}
  \end{flushright}
\end{minipage}

\vfill

% Bottom of the page
{\large Version 0.9.1 du\\ \today}

\end{center}
\end{titlepage}

%%%%%%%%%%%%%%%%%%%%%%%%%%%%%
%%% Remerciments %%%
%%%%%%%%%%%%%%%%%%%%%%%%%%%%%

\frontmatter

%\chapter*{Remerciements}
%Je tiens à remercier toutes les personnes qui ont contribué au succès de mon stage et qui m'ont aidé lors de la rédaction de ce rapport.

%Tout d'abord, j'adresse mes remerciements à mon professeur, Mme.~Sandrine \textsc{Lanquetin} de l'Université de Bourgogne qui m'a permis de postuler dans cette entreprise.

%Je tiens à remercier vivement mon maitre de stage, Mr.~Olivier \textsc{Leick}, responsable solution au sein de l'entreprise Orange, pour son accueil, ses conseils, le temps passé ensemble et le partage de son expertise au quotidien ainsi que ses précieux conseils lors de la rédaction de ce rapport.

%Je remercie également toute l'équipe de la SDFY Orange pour leur accueil et leur esprit d'équipe.

%Enfin, je tiens à remercier toutes les personnes qui m'ont conseillé et relu lors de la rédaction de ce rapport de stage : ma famille, ma compagne Mélane, et mon amie Nilu.$

\clearpage
\tableofcontents

\clearpage
\listoffigures

\clearpage

%%%%%%%%%%%%%%%%%%%%%%%%%%%%%%%%%%%%%%%%%%%%%
%                 Accronymes                %
%%%%%%%%%%%%%%%%%%%%%%%%%%%%%%%%%%%%%%%%%%%%%
\chapter*{Liste des sigles et acronymes}

Liste des sigles et acronymes internes :\\

\begin{acronym}[CP-OFDMX]

\acro{DFY}{\emph{Data Factory}}
\acro{OKAPI}{\emph{Oauth2 Kerberos API}}
\acro{PnS}{\emph{Profile and Syndication}}
\acro{SDFY}{\emph{Smart Data Factory}}
\acro{SI}{\emph{Système d'Informations}}


\end{acronym}


Liste des sigles et acronymes techniques :\\

\begin{acronym}[CP-OFDMX]

\acro{API}{\emph{Application Programming Interface - Interface de programmation}}
\acro{SSO}{\emph{Signe Sign-On - Authentification unique}}
\acro{TTL}{\emph{Time To Live - Durée de vie }}

\end{acronym}

%%%%%%%%%%%%%%%%%%%%%%%%%%%%%%%%%%%%%%%%%%%%
%%% Content of the report and references %%%
%%%%%%%%%%%%%%%%%%%%%%%%%%%%%%%%%%%%%%%%%%%%

\mainmatter
\pagestyle{fancy}

\cleardoublepage

\chapter*{Introduction}
\addcontentsline{toc}{chapter}{Introduction}
\markboth{Introduction}{Introduction}
\label{chap:introduction}
%\minitoc

Avec 30 millions de clients mobile et plus de 18 millions de clients  pour le haut et très haut débit fixe, Orange est le premier opérateur en France et la 50\up{ième} entreprise la plus influente au monde en 2016. Avec une telle importance autant au niveau national qu'international, la firme se doit de rester innovante et à la pointe de la technologie pour pouvoir répondre aux besoins de plus en plus précis et diversifiés de leurs clients. La division que j’ai intégré, la Smart Data Factory (SDFY), opère sur plusieurs secteurs différents allant de la gestion de base de données à la recommandation personnalisée en passant par les application mobiles et la sécurité. Avec une telle diversité de secteurs d’activités, il n’est pas évident de tenir toutes les équipes au courant de toutes les nouveautés et avancées au sein de chaque projet ni de pouvoir les expliquerez simplement et ludiquement aux personnes sans connaissances techniques.\\
L’offre de stage à laquelle j’ai répondu s’inscrit dans la lignée d’un projet qui avait été initié par un prestataire externe : pouvoir expliquer ludiquement et simplement sous forme de vidéos le fonctionnement et l’intérêt des nouvelles  interfaces de programmations (APIs) produites au sein de la SFDY dans le cardre du programme PnS.com, qui a pour but de fournir un stockage des données en libre service. J’ai choisi de répondre à cette offre car j’étais intéressé par le côté créatif de la création de vidéos qui était une suite directe du cours de « Pratiques plastiques » que j’ai suivi cette année et qui m’avait beaucoup plu. C’était aussi l’occasion pour moi de mettre en pratique les connaissances techniques acquises au cours de mes cinq années de Master à l’Université de Bourgogne.\\
Dans ce rapport, nous étudierons plus en détail l’entreprise et son secteur d’activité, puis nous nous concentrerons sur mon stage, en nous focalisant sur mes différentes missions puis mon bilan.
%%% Local Variables: 
%%% mode: latex
%%% TeX-master: "isae-report-template"
%%% End: 

\chapter{L'entreprise et son secteur d'activité}
\label{chap:premierchapitre}

\section{L'histoire d'Orange}


En juillet 1992, le premier opérateur mobile  de France voit le jour sous le nom Itineris. Deux ans plus tard Orange est lancé sur le marché britannique et deviens le quatrième opérateur du pays. Orange est côté à la bourse de Londres et au NASDAQ en avril 1996. A ce moment, Orange possède environ 500 000 abonnés et un mois plus tard figure dans la liste des plus grandes entreprises britanniques. A la fin de l'année 1998 Orange compte plus d'1 million d'abonnés et à lancé une promesse de performance réseau. Au même moment, en France, Itineris couvre 97\% de la population avec 7 700 relais et compte 5,5 millions de clients. Un an plus tard, après avoir lancé le premier système de reconnaissance vocale mobile du Royaume Uni, Orange esr racheté par Mannesmann ce qui porte le nombre d'abonnés à plus de 3,5 millions. Quelques mois plus tard, Orange lance sa première plate forme internet mobile : "orange.net". En février 2000, alors que Mannesmann venait d'être racheté, il est entrepris de vendre Orange qui sera racheté un mois plus tard par France Télécom. Les opérations de téléphonies mobiles de France Télécom sont donc fusionnées avec celles d'Orange, ce qui pousse le nouveau groupe "Orange S.A" \textit{Société Anonyme)} à être présent dans 20 pays. Un an plus tard, en juin 2001, mobicarte, Ola et Itineris, qui comple alors plus de 10 millions de clients, fusionnent pour devenir le Orange que nous connaissons. En juin 2006, Orange rachète Wanadoo et Ma Ligne TV. En 2011, France Télécom communique en tant que "Groupe France Télécom - Orange" puis vote le changement de nom pour "Orange" en mail 2013, nom qui sera adopté et effectif en 2013.



\section{Secteur d'activité}

Avec la démocratisation d'internet et des objets connectés, nous sommes de plus en plus connectés. Avec plus de 3,5 milliards d'internet, chiffre en perpétuelle hausse, et plus de 3h par jour passées devant l'écran de son téléphone, les milieux de la téléphonie et de l'internet doivent toujours être à l'écoute des besoins de ses clients pour pouvoir répondre à leurs besoins en constante évolution. Orange se donne pour but de répondre à chaque besoin, que cela soit une connectivité sans faille ou bien encore proposer un bon rapport qualité/prix dans le but de proposer à chaque client la meilleure expérience. A plus long terme, Orange souhaite faire vivre cette expérience à ses clients au quotidien en concevant des services numériques leur permettant de profiter de ce qui leur est essentiel. \\
Les principales activités d'Orange sont :\\
Une offre d'accès internet en haute débit, ADSL et fibre optique et des services multimédias via la Livebox, tels que la TV ou un système de vidéos à la demande. A ce jour, Orange compte 3,3 millions de clients de la fibre et plus de 11 millions pour le haut débit.\\
Orange fournit aussi des services mobiles sur des réseaux de 2, 3 et 4\up{ème} génération avec plus de 30 millions d'utilisateurs, ce qui place Orange 1\up{er} opérateur mobile en France\\
Orange propose aussi des applications permettant des transactions financières, "Orange Money", comptant plus de 29 millions de clients dans 17 pays. En avril dernier à été lancé Orange Bank, un service ayant pour ambition d'être une banque mobile.


\section{La Digital Dactory}



%%% Local Variables: 
%%% mode: latex
%%% TeX-master: "isae-report-template"
%%% End: 
\chapter{Mon stage}
\label{sec:unchapitre}

Lorsque j'ai répondu à l'offre de stage, il été avait initialement prévu que je produise une newsletter ainsi que trois vidéos sur trois APIs différentes qui sont :

\begin{itemize}
    \item Gat'Ape/Okapi
    \item Zbus
    \item Valkey
\end{itemize}

Cependant, aux vues de mon avancement et de l'intérêt suscité par les vidéos produites, j'ai accepté de travailler sur trois autres projets qui sont :

\begin{itemize}
    \item Explication des difficultés liées à la création et à la publication d'APIs avant l'arrivée du cloud au sein d'Orange
    \item Présentation des avantages du travail en DevOps
    \item Promotion de PnS.com
\end{itemize}

Les vidéos traitant des APIs se devaient d'être vulgarisées et simple à comprendre étant donné qu'elles pourraient être destinés à un public très vaste. Chaque vidéo se doit d'être compréhensible par un programmeur, un manager ou un commercial par exemple. La principale difficulté est de devoir vulgariser un maximum tout en gardant un maximum de précision pour qu'elles restent pertinentes pour les personnes les plus techniques. La diffusion de ces vidéos peut se faire pour montrer les dernières avancées au sein de la SDFY, ou encore pour promouvoir un produit, ou le proposer à des clients en tant que nouvelle solution. \\

En revanche, l'approche pour les 3 autres vidéos est totalement différente, elles s'adressent à un public ciblé et ont principalement un but promotionnel. Deux de ces vidéos ont été réalisées pour un événement particulier. \\

Avant de développer développer le contexte et la production de chaque vidéo, je vais d'abord présenter le programme PnS.com et ses différents composants. 

\section{Présentattion générale de l'écosystème PnS.com}


\section{Vidéo sur Gat'Ape/Okapi}

\subsection{Contexte}
Gat'ape/Okapi est une API faisant partie de l'écosystème PnS.com qui permet d'exposer, de s'autentifier et de sécuriser les APIs de cet écosystème. Gat'ape et Okapi ont deux fonctions bien distintes.\\

Okapi est le système d'authentification qui va permettre la connexion aux autres APIs. Okapi est basé sur le couple Oauth2/Kerberos qui va permettre une identité indépendante à chaque entité tout en offrant un sytème d'authentification unique (SSO) qui va permettre à l'utilisateur d'accéder à plusieurs applications en ne s'authentifiant qu'une seule fois. Cette authentification repose sur un système de clé secrète et de jeton. Il est à noter que des options de sécurité plus complexes sont aussi disponibles.\\

Gat'ape quant à elle est la passerelle qui va permettre d'exposer les APIs, c'est à dire, des les rendre visible à d'autres utilisateurs pour qu'ils puissent s'y connecter. Gat'ape va pouvoir offir un conrôle d'accès et un contrôle de traffic permettant de réguler le flux des utilisateurs. Gat'ape est également responsable  de l'authentification et de la sécutité lors de la consommation par Okapi. En plus de cela, gat'ape est également scalable, c'est à dire qu'elle va pouvoir maintenir son activité et sa performance même lors d'une forte demande.\\

Dans les faits la connexion grâce à Gat'ape/okapi se passe comme suit : 
L'application qui vient se connecter possède une clé secrète. Cette clé secrète est envoyée à Okapi. Lors du traitement, Okapi va renvoyer un jeton d'authentification unique au consommateur ainsi qu'un digest. Ceci est la phase d'authentification.  Ce digest est ensuite envoyé du consommateur vers Gat'ape. Si le digest est valide, le consommateur va envoyer le jeton précédemment obtenu vers l'API à laquelle il souhaitait se connecter. L'API va à son tour vérifier le token puis le renvoyer ainsi que les informations demandées par le consommateur. Ces informations luis seront transmises via Gat'ape. Si une de ces vérifications de token échoue, le protocole recommence un envoi de token. L'opération peut ainsi être réitérée un certain nombre de fois sans avoir à se ré authentifier. Cette durée de vie (TTL) est paramétrable.

\begin{figure}[htp]
  \centering
  \includegraphics[width=15cm]{images/gao/gao1}
  \caption{Fonctionnement de Gat'Ape/Okapi.}
  \label{fig:une-autre-image}
\end{figure}



\subsection{Production}
Avant de commencer la production à proprement parlé j'ai tout d'abord du lire la documentation relative à cette API pour en comprendre l'utilité et le fonctionnement. Une fois que la fonction et le fonctionnement de cette API m'était plus clair, j'ai pris rendez vous avec le responsable solution de cette API. Nous avons fait une réunion dans laquelle je lui ai demandé les points qu'il voulait mettre en avant grâce à la vidéo et les principales innovations qu'apportait Gat'ape/okapi par rapport à l'ancienne solution. Une fois d'accord sur les points à aborder je me suis mis à produire un script, et un storyboard qui sont respectivement le texte de la vidéo et les images et animations clefs de celle ci. Une fois le script et le storyboard finalisé, j'ai de nouveau rencontré le responsable de Gat'ape/Okapi pour les lui proposer. Une fois ces documents validés, j'ai pu me lancer dans la production à proprement parler. J'ai choisi de créer un thème sobre et direct pour que les spectateurs ne se perdent pas dans les détails. Un fond blanc et les illustrations trouvées sur le site de la marque. De cette manière j'étais sûr de respecter les conditions de la charte graphique, à savoir, au mois 20\% de couleur orange sur les illustrations ainsi que les couleurs et la police officielle.\\

Au lancement de la vidéo, les logos de Gat'ape et Okapi arrivent depuis chaque côté de l'écran, présentant ainsi les deux APIs. La partie suivante de la vidéo explique brièvement la fonction générale de Gat'ape/Okapi. Les APIs exposés au travers de Gat'ape apparaissent en premier, pour symboliser le fait qu'elles sont déjà présentes et visibles, puis les consommateurs font leur apparition et ces derniers sont reliés aux APIs via des flèches en passant par le duo Gat'ape/Okapi. Vient ensuite l'explication de l'identification. Le nom du protocole d'authentification, Okapi, apparaît et est décomposé pour expliquer son acronyme. Dans un premier temps le protocole Oauth2 est expliqué comme étant le moyen de mise en relation du consommateur avec Okapi,  puis Kerberos est succinctement expliqué comme étant un protocole d'identification à base de clé secrète et de jeton. Kerberos étant crucial et complexe, j'ai choisi de l'expliciter dans la partie suivante de la vidéo autour d'une situation connue de tous, l'accès à une salle de cinéma. Dans un premier temps le spectateur, dans le cas concret : le consommateur, va récupérer au guichet (Okapi) son ticket. Une fois ceci fait, le spectateur doit passer devant l'ouvreur au point d'accès de la salle qui va déchirer le ticket et en garder une partie. Lors du contrôle (l'accès à Gat'ape), il est vérifié que les deux morceaux de tickets vont bien ensemble, si c'est le cas le spectateur peut accéder à sa séance (Le consommateur peut accéder à l'API). Il est ensuite précisé que le ticket à une durée bien définie, une séance dans le cas d'un ticket de cinéma, un Time To Live (durée de vie) paramétrable dans le cas de Gat'ape/Okapi. Il est ensuite mis en avant les possibilités qu'offre Gat'ape dans la mise en relation entre le consommateur et l'API, tels le contrôle d'accès, qui va permettre d'autoriser ou non l'accès à une API à certains consommateurs et le contrôle de trafic qui va permettre de réguler les flux d'accès en fonction de la demande des consommateurs et de la disponibilité de l'api. Pour représenter le contrôle d'accès j'ai choisi d'utiliser des croix rouges et des coches vertes afin de rester simple et compréhensible. J'ai choisi d'illustre le contrôle de trafic avec des feux tricolore que l'on voit changer de couleur pour symboliser le fait que ce contrôle n'est pas figé et se déroule en continu. La fonctionnalité mise en avant ensuite est la détection d'erreurs et d'anomalies qui est fournie par une autre API intégrée à Gat'ape, représentée par un moniteur qui contient une sorte d'électrocardiogramme qui grandit dangereusement à un moment, cette anomalie est ensuite entourée en rouge avec un panneau danger, qui représente la détection. Les fonctionnalité mentionnées ensuite sont un surcoût de temps de réponse quasi nul, qui n'impacte pas les performances, et une architecture tri site, qui assure  une très haute disponibilité. En effet, si l'un des trois site venait à subir une maintenance, une attaque ou être indisponible pour une autre raison, il y a deux autres sites qui pourront fournir l'authentification et garantir la sécurité aux consommateurs. J'ai ensuite choisi de faire un résumé des principales informations car la vidéo était un peu longue et assez détaillée sur certains points, et les personnes les moins techniques pourraient avoir décroché. Il est donc rappelé que Gat'ape permet à des APIs d'être exposées à des consommateurs qui pourront s'y connecter de manière sécurisée efficace, via l'apparition de deux logos rappelant ces informations. Il est ensuite indiqué via les logos respectifs qu'il existe un SDK pour permettre l'intégration facile de Gat'ape/Okapi à son application et qu'il existe un modèle d'API contenant Gat'ape/okapi prêt à l'usage pour les personnes souhaitant développer une API. Dans la dernière séquence, sont indiqués les liens vers la documentation et le mail de l'équipe en charge pour les personnes souhaitant avoir plus d'informations à ce sijet. 



\section{Vidéo sur Zbus}

\subsection{Contexte}

\subsection{Production}



\section{Vidép sur Valkey}

\subsection{Contexte}

\subsection{Production}

%%% Local Variables: 
%%% mode: latex
%%% TeX-master: "isae-report-template"
%%% End: 
\chapter*{Conclusion et perspectives}
\addcontentsline{toc}{chapter}{Conclusion}
\markboth{Conclusion}{Conclusion}
\label{sec:conclusion}

Au long de ces six mois de stage passés sur le site d'Orange de Sophia-Antipolis, j'ai pu découvrir un métier aux enjeux variés allant de l'explication de technologies en interne à 
l'explication de méthodes de travail qui ciblera des personnes en dehors de l'entreprise. 
J'ai particulièrement apprécié la liberté de création qui m'a été accordée pour la réalisation des vidéos, même si j'aurais aimé bénéficier de plus de temps pour pouvoir créer moi même la totalité des illustrations. J'ai aussi dû approfondir mes connaissances du logiciel After Effect utilisé pour le montage pour pouvoir produire des effets de transitions que nous n'avions pas eu le temps de réaliser dans les cours de pratiques plastiques. Ce stage m'aura permis de confirmer que les missions créatives sont les plus adaptées pour moi, mais qu'il ne faut pas non plus négliger l'aspect technique qui reste très présent.\\

La DFY m’a accueilli pendant une période charnière : son passage aux APIs via le cloud, et je suis fier d’avoir pu prendre part à ce changement qui va permettre une facilité d'intégration des applications et l'accès aux données ainsi qu'à la sécurisation des échanges qui est un paramètre plus qu'important de nos jours. J'ai notamment étoffé mes connaissances sur la sécurité et les étapes de créations d'une application qui change ne nos habitudes universitaires. En effet là où nous commençons chaque application à partir de rien, celles de la DFY sont construites à partir de briques pré-établies. Les connaissances acquises au cours de mon cursus m'ont permis de comprendre le fonctionnement de la plupart des composantes de l'écosystème PnS.\\

Suite à cette expérience, je souhaite m'orienter dans un secteur qui soit aussi technologique et innovant que celui dans lequel j'ai effectué ce stage et dans lequel il y aurait toujours une part importante de créativité.


%%% Local Variables: 
%%% mode: latex
%%% TeX-master: "isae-report-template"
%%% End: 



\appendix

\bibliographystyle{authoryear-fr}
\bibliography{references}

\clearpage

%%%%%%%%%%%%%%%%
%%% Abstract %%%
%%%%%%%%%%%%%%%%

\thispagestyle{empty}

\vspace*{\fill}
\noindent\rule[2pt]{\textwidth}{0.5pt}\\


\noindent\rule[2pt]{\textwidth}{0.5pt}
\begin{center}
  Orange\\
  790 Avenue Maurice Donat\\
  06250 Mougins
\end{center}
\vspace*{\fill}

\end{document}
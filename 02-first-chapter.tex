\chapter{L'entreprise et son secteur d'activité}
\label{chap:premierchapitre}

\section{L'histoire d'Orange}


En juillet 1992, le premier opérateur mobile  de France voit le jour sous le nom Itineris. Deux ans plus tard Orange est lancée sur le marché britannique et devient le quatrième opérateur du pays. Orange est côté à la bourse de Londres et au NASDAQ en avril 1996. A ce moment, Orange possède environ 500 000 abonnés et un mois plus tard figure dans la liste des plus grandes entreprises britanniques. À la fin de l'année 1998 Orange compte plus d'1 million d'abonnés et a lancé une promesse de performance réseau. Au même moment, en France, Itineris couvre 97\% de la population avec 7 700 relais et compte 5,5 millions de clients. Un an plus tard, après avoir lancé le premier système de reconnaissance vocale mobile du Royaume Uni, Orange est racheteé par Mannesmann ce qui porte le nombre d'abonnés à plus de 3,5 millions. Quelques mois plus tard, Orange lance sa première plate forme internet mobile : "orange.net". En février 2000, alors que Mannesmann venait d'être rachetée, il est entrepris de vendre Orange qui sera rachetée un mois plus tard par France Télécom. Les opérations de téléphonies mobiles de France Télécom sont donc fusionnées avec celles d'Orange, ce qui pousse le nouveau groupe "Orange S.A" \textit{(Société Anonyme)} à être présent dans 20 pays. Un an plus tard, en juin 2001, mobicarte, Ola et Itineris, qui compte alors plus de 10 millions de clients, fusionnent pour devenir le Orange que nous connaissons. En juin 2006, Orange rachète Wanadoo. En 2011, France Télécom communique en tant que "Groupe France Télécom - Orange" puis vote le changement de nom pour "Orange" en mai 2013.



\section{Secteur d'activité}

Avec la démocratisation d'internet et des objets reliés au web, nous sommes de plus en plus connectés. Avec plus de 3,5 milliards d'internautes, chiffre en perpétuelle hausse, et plus de 3h par jour passées devant l'écran de son téléphone, les milieux de la téléphonie et de l'internet doivent toujours être à l'écoute des besoins de ses clients pour pouvoir répondre à leurs demandes en constante évolution. Orange se donne pour but de répondre à chaque besoin, que cela soit une connectivité sans faille ou bien encore conseiller un bon rapport qualité/prix dans le but de proposer à chaque client la meilleure expérience. À plus long terme, Orange souhaite faire vivre cette expérience à ses clients au quotidien en concevant des services numériques leur permettant de profiter de ce qui leur est essentiel. \\
Les principales activités d'Orange sont :\\
Une offre d'accès internet en haute débit, ADSL et fibre optique et des services multimédias via la Livebox, tels que la TV ou un système de vidéos à la demande. A ce jour, Orange compte 3,3 millions de clients de la fibre et plus de 11 millions pour le haut débit.\\
Orange fournit aussi des services mobiles sur des réseaux de 2, 3 et 4\up{ème} génération avec plus de 30 millions d'utilisateurs, ce qui place Orange 1\up{er} opérateur mobile en France.\\
Orange propose aussi des applications permettant des transactions financières, "Orange Money", comptant plus de 29 millions de clients dans 17 pays. En avril dernier à été lancé Orange Bank, un service ayant pour ambition d'être une banque mobile.


\section{La Digital Factory}

Au sein du groupe se trouve la Digital Factory (DFY) qui est une division ayant pour buts de développer et suivre de bout en bout des projets digitaux multi supports avec une grande capacité à innover et en se servant de technologies open-source et réutilisables telles que des interfaces de programmations (APIs). Pour se faire, quatre principes sont mis en \oe{}uvre à la DFY :

\begin{itemize}
    \item Chercher à réduire la taille unitaire des projets, en itérant plusieurs fois avant 	d’atteindre l’objectif final
    \item Organiser les équipes par produit avec au moins un représentant métier
    \item Privilégier l’autonomie et la responsabilité des équipes
    \item Rechercher la proximité avec le marketing, en partageant les mêmes objectifs business et techniques
\end{itemize}

De manière plus précise, la DFY est en charge de développer et de maintenir les services du portail Orange, comprenant entre autre le site orange.fr, les chaînes, les applications mail,  les logiciels développés par Orange sur ordinateur.
Elle est également responsable de domaines plus sensibles comme la sécurité, les projets ainsi que la gestion et le traitement des données.\\

La DFY regroupe en son sein 400 développeurs pour 8 salles d’hébergement comprenant 3200 serveurs traitant 21 Gb de données par seconde et 350 000 requêtes http par seconde en pic provoquées par plus de 25 millions de visiteurs par jour sur le portail Orange.

Cette entité regroupe plusieurs centres de compétences telles que le développement tous écrans et périphériques, l'ergonomie et design, de l'hébergement et supervision, le data science, la sécurité et la gestion de projet. Tous ces projets étant gérés de manière agile et en DevOps. 

%%% Local Variables: 
%%% mode: latex
%%% TeX-master: "isae-report-template"
%%% End: 